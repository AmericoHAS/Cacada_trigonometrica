\documentclass[a4paper,12pt]{article}
\usepackage[margin=2cm]{geometry}
\usepackage[brazil]{babel}
\usepackage[utf8]{inputenc}
\usepackage{amsmath, amssymb}
\usepackage{graphicx}
\usepackage{multicol}
\setlength{\columnsep}{1cm}

\title{\textbf{Caçada Trigonométrica – Exercícios das Estações}}
\author{Professor: Haward Antunny da Silva Américo}
\date{}

\begin{document}
	\maketitle
	
	\begin{multicols}{2}
		
		\section*{Estação 1 – Mesa de Ping-Pong}
		Encontre a largura da mesa de ping-pong (altura do triângulo formado com giz).  
		Sabendo que a mesa é dividida ao meio ($m=n$) e que $n$ é informado no local.  
		Use a relação:  
		\[
		h^2 = m \cdot n
		\]
		
		\vspace{0.5cm}
		
		\section*{Estação 2 – Placa Encostada no Muro}
		Uma placa está encostada em um muro, formando um triângulo retângulo com o chão.  
		A base da placa até o muro mede $3$ m e a altura do muro é informada no local.  
		Calcule o comprimento da placa utilizando Pitágoras.
		
		\vspace{0.5cm}
		
		\section*{Estação 3 – Retângulo na Parede}
		Na parede há um retângulo $ABCD$. Pontos $D$ e $B$ projetam sobre a diagonal $AC$ por segmentos perpendiculares $DE$ e $BF$.  
		Sabendo $AB$ e $BC$ (informados no local), determine a medida aproximada de $EF$.
		
		\vspace{0.5cm}
		
		\section*{Estação 4 – Caminho mais Curto}
		Uma pessoa caminha no pátio, contornando obstáculos, passando pelos segmentos:  
		$BC$, $DE$ e $EF$.  
		Sabendo as medidas de cada segmento (informadas no local), determine a distância mínima entre $A$ e $F$:
		\[
		AF = \sqrt{(BC + DE)^2 + EF^2}
		\]
		
		\vspace{0.5cm}
		
		\section*{Estação 5 – Rede de Vôlei}
		Uma corda é esticada do topo do mastro da rede até o chão.  
		A altura do mastro é $2,55$ m e a distância até a base da corda (no chão) é informada no local.  
		Calcule o comprimento da corda usando Pitágoras.
		
		\vspace{0.5cm}
		
		\section*{Estação 6 – Escada da Escola}
		Uma escada com 8 degraus de mesma altura precisa de reforma.  
		O corrimão será trocado, sabendo que a altura do corrimão é de $90$ cm e a distância de cada degrau é informada no local.  
		Calcule o comprimento do corrimão.
		
		\vspace{0.5cm}
		
		\section*{Estação 7 – Poste com Duas Cordas}
		Um poste está preso ao chão por duas cordas.  
		As medidas das cordas são informadas no local.  
		Determine a distância entre os pontos em que as cordas tocam o chão (base do triângulo formado).
		
		\vspace{0.5cm}
		
		\section*{Estação 8 – Mastro da Bandeira}
		De um ponto fixo no chão, observa-se o topo de um mastro sob um ângulo informado no local.  
		A distância até o mastro é de $5$ m.  
		Determine a altura aproximada do mastro usando:
		\[
		h = d \cdot \tan(\theta)
		\]
		
		\vspace{0.5cm}
		
		\section*{Estação 9 – Retângulo de Ferro na Parede}
		Um barbante foi esticado na diagonal de um retângulo.  
		Sabendo o valor da diagonal (informado no local) e o ângulo medido com transferidor, calcule a área do retângulo:  
	
	\end{multicols}
	
\end{document}
